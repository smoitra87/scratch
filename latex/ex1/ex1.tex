\documentclass{article}[12pt]
\usepackage{amsmath,amssymb,amsthm}
\usepackage{graphicx}

\numberwithin{equation}{section}

\begin{document}

\title{A simple Title}
\author{Subhodeep}

\maketitle

\begin{abstract}
This is the abstract
\end{abstract}

\section{Ex1 section}

The first align using splits is referenced by ref \ref{eq1}. It is also possible to use eqref \eqref{eq1}. 

\begin{equation}
\label{eq1} 
\begin{split}
x_1 = a_1 + b_1\\
 = \alpha_1 + \beta_1 + \gamma_1\\
x_2 = a_2 + b_2
\end{split}
\end{equation}

The above equations do not align properly in the split mode. To make things align properly you have to use the \& operator. and it then gets solved

\begin{equation}
\begin{split}
x_1 & = a_1 + b_1\\
 & = \alpha_1 + \beta_1 + \gamma_1\\
x_2 & = a_2 + b_2
\end{split}
\end{equation}

We will no use multiline to just wrap over multiple lines


Did not work unfortunately, :(
%\begin{equation}
%\begin{multiline}
%a + b + c+d+e+f \\
%x+y+z+\alpha
%\end{multiline}
%\end{equation}

This is an example of inline math $x=1$ and \(\beta\)


Now, we will look at an examples of align.

\begin{align}
x_1 = a_1 + b_1\\
 = \alpha_1 + \beta_1 + \gamma_1\\
x_2 = a_2 + b_2
\end{align}

Next we will properly align it


\begin{align}
x_1 & = a_1 + b_1\\
 & = \alpha_1 + \beta_1 + \gamma_1\\
x_2 &  = a_2 + b_2
\end{align}


It is possible to use align to align just about anything. Basically \& acts like a tabspace and all the contents are arranged around that. 


\begin{align}
x_1 & = a_1 + b_1 & x_2 &  = a_2 + b_2 \\
x_1 & > a_1 + b_1 & x_2 &  < a_2 + b_2
\end{align}

Note that all the equations are numbered when using align

Another possible thing to try is to name the equations based on the section number. This can be achieved by making a numberswithin preprocessor directive


\end{document}






