\documentclass{article}[12pt]
\usepackage{amsmath,amssymb,amsthm}
\usepackage{graphicx}
\usepackage[margin=1in]{geometry}

\begin{document}

\title{Derivation of Sigmoid using Tanh}
\author{Subhodeep Moitra \\ \tt{subhodee@cs.cmu.edu}}

\maketitle

\section{Derivation of sigmoid in terms of tanh}
Sigmoid functions are ubiquitous in neural network. Did you know that they can be calculated using tanh functions exactly ? Here's how


\begin{align*}
\sinh(x) &= e^x - e^{-x} \\
\cosh(x) &= e^x + e^{-x} \\
\tanh(x) &= \frac{e^x - e^{-x}}{e^x + e^{-x}}  \\
	&= \frac{e^{2x} - 1}{e^{2x} + 1}  \\
\end{align*}


Using the above results we can represent the sigmoid function $\sigma(x)$ as a function of $\tanh(x)$

\begin{align*}
\sigma(x) &= \frac{1}{1 + e^{-x}} \\
		&= \frac{e^x}{1 + e^{x}} \\
		&= \frac{1}{2}\left(\frac{2e^x}{e^x + 1}\right) \\
		&= \frac{1}{2}\left(\frac{e^{x} - 1}{e^{x} + 1} + 1\right) \\
		&= \frac{1 + \tanh(x/2)}{2} 
\end{align*}

\end{document}






